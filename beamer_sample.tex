%xelatex

\documentclass{beamer}
\usepackage[utf8]{inputenc}
\usepackage[english]{babel}
\usepackage{amssymb}
\usepackage{cite,natbib} %For better bibliography.
\usepackage{graphicx}
\usepackage{mathtools}
\usepackage{tikz-cd} %For commutative diagrams.
\usetikzlibrary{matrix}

\usetheme[block=fill,numbering=fraction,progressbar=frametitle]{metropolis}
\useinnertheme{circles}

%\setbeamertemplate{theorems}[numbered]

\usepackage{thmtools} %for spacing of theroem heading in title environment.
\declaretheoremstyle[
headfont=\bfseries,%
headpunct={\vspace{\topsep}\newline}, %
numbered=no,
spaceabove=3\topsep, %
postheadspace=0 pt ]{exercs}


%For better bibliography.
\bibliographystyle{unsrt}
\renewcommand{\bibfont}{\footnotesize}
\setbeamertemplate{frametitle continuation}[from second]

\newcommand{\bbox}[1]{
	\noindent\fbox{ \parbox{\textwidth}{#1} \vspace{2pt}} }
\renewcommand{\bibsection}{}

%Theorem environments
\usepackage{mdframed}
\mdfsetup{skipabove=1em,skipbelow=0em}
\theoremstyle{definition}
\newmdtheoremenv[nobreak=true]{lmma}{Lemma}
\newmdtheoremenv{conclusion}{Conclusion}
\newtheorem{defn}{Definition}
\newmdtheoremenv{prop}{Proposition}
\newmdtheoremenv{lma}{Lemma}
\newmdtheoremenv{cor}{Corollary}
\newmdtheoremenv{thm}{Theorem}
\newtheorem*{obs}{Observation}
\theoremstyle{remark}
\newtheorem*{rmk}{Remark}

% Some shortcuts
\newcommand\N{\ensuremath{\mathbb{N}}}
\newcommand\R{\ensuremath{\mathbb{R}}}
\newcommand\Z{\ensuremath{\mathbb{Z}}}
\renewcommand\O{\ensuremath{\emptyset}}
\newcommand\Q{\ensuremath{\mathbb{Q}}}
\renewcommand\C{\ensuremath{\mathbb{C}}}
\newcommand\F{\ensuremath{\mathbb{F}}}

%Make implies and impliedby shorter
\let\implies\Rightarrow
\let\impliedby\Leftarrow
\let\iff\Leftrightarrow
\let\epsilon\varepsilon
\let\phi\varphi

% Add \contra symbol to denote contradiction
\usepackage{stmaryrd} % for \lightning
\newcommand\contra{\scalebox{1.5}{$\lightning$}}

\usepackage{thmtools}
\usepackage{kantlipsum}

\declaretheoremstyle[
headfont=\bfseries,%
headpunct={\vspace{\topsep}\newline}, %
numbered=no,
spaceabove=3\topsep, %
postheadspace=0 pt ]{exercs}
\declaretheorem[name=EXERCISES,style=exercs]{problems}

\author{Akshay Dhan}

%-------------------------------------------------------------
\usecolortheme{dove} %lily, spruce, wolverine, monarca, sidebartab,dove,seagull

\title{Title}
\subtitle{Subtitle}
\institute{Institute}
\date{\today}
%\titlegraphic{\hfill\includegraphics[height=0.5cm]{Images/logo.png}}


\begin{document}
%-------------------------------------------------------------
\frame{\titlepage}
%-------------------------------------------------------------
\begin{frame}{lorem ipsum}
Lorem ipsum dolor sit amet, consetetur sadipscing elitr, sed diam nonumy eirmod
	tempor invidunt ut labore et dolore magna aliquyam erat, sed diam
	voluptua. At vero eos et accusam et justo duo dolores et ea rebum. Stet
	clita kasd gubergren, no sea takimata sanctus est Lorem ipsum dolor sit
	amet.
\end{frame}
%-------------------------------------------------------------
\begin{frame}{lorem ipsum}
	\begin{thm}[lorem]
Lorem ipsum dolor sit amet, consetetur sadipscing elitr, sed diam nonumy
		eirmod tempor invidunt ut labore et dolore magna aliquyam
		erat, sed diam voluptua. At vero eos et accusam et justo
		duo dolores et ea rebum. Stet clita kasd gubergren, no sea
		takimata sanctus est Lorem ipsum dolor sit amet.
\end{thm}
\end{frame}
%-------------------------------------------------------------

\begin{frame}[t]{References}\vspace{10pt}
\begin{thebibliography}{abcd}
\bibitem{AB}
A. Author Name, 
{\it Title of Paper}, Journal Name, Volume No(Year), (page nos) 1-2.

\bibitem{BZ1}
R. Taylor, A. Wiles, {\it Ring theoretic properties of certain Hecke algebras}, Ann. of Math., \textbf{141}(3) (1995), 553-572.

\end{thebibliography}
\end{frame}
%-------------------------------------------------------------

\end{document}
